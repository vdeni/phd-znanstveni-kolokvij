\documentclass[aspectratio=169]{beamer}

\usepackage[croatian]{babel}
\usepackage[T1]{fontenc}
\usepackage[utf8]{inputenc}
\usepackage{lmodern}
\usepackage{microtype}
\usepackage{csquotes}
\usepackage[natbibapa]{apacite}
\usepackage{url}
\usepackage{fontawesome}

%%%%% titlepage
\title[]{\textit{Megastudy} paradigma i procesiranje jezika}
\subtitle{\footnotesize Izlaganje za Znanstveni kolokvij}
\author[]{Denis Vlašiček \\[5em]
    \scriptsize
    \url{https://osf.io/zqa7y/}
}
\date[]{\footnotesize \today}
%%%%%

%%%%% prilagodba referenci na hrvatski
\renewcommand{\BCBT}{}
\renewcommand{\BCBL}{}%  comma before last author when no. of authors > 2
\renewcommand{\BOthers}[1]{i sur.\hbox{}}% ``and others''
\renewcommand{\BBAA}{i}
\renewcommand{\BBAB}{i}
\renewcommand{\BIn}{U:}
\renewcommand{\BED}{Ur.\hbox{}}
\renewcommand{\BEDS}{Ur.\hbox{}}
\renewcommand{\BPGS}{str.\hbox{}}
\renewcommand{\BRetrievedFrom}{Preuzeto s:\hbox{}}
\renewcommand{\BEd}{izdanje}
\renewcommand{\BVOL}{svezak}
%%%%%

%%%%% beamer tema
\usetheme{CambridgeUS}

\usecolortheme{dove}

\setbeamertemplate{itemize items}[default]

\setbeamertemplate{itemize item}{\textbullet}
\setbeamertemplate{itemize subitem}{\(\triangleright\)}

\setbeamertemplate{section page}{
    \begin{centering}
        \Large
        \insertsection\par
    \end{centering}
}

%%%%% serif font kao default
\usefonttheme{serif}
%%%%%

\setbeamertemplate{navigation symbols}{}

\makeatletter
\newenvironment{noheadline}{
    \setbeamertemplate{headline}{}
    \addtobeamertemplate{frametitle}{\vspace*{-0.9\baselineskip}}{}
}{}
\makeatother
%%%%%

%%%%% nastavljanje enumerate
\newcounter{saveenumi}
\newcommand{\seti}{\setcounter{saveenumi}{\value{enumi}}}
\newcommand{\conti}{\setcounter{enumi}{\value{saveenumi}}}

\resetcounteronoverlays{saveenumi}
%%%%%

%%%%5 fontsize za bibliografiju
\renewcommand*{\bibfont}{\scriptsize}
%%%%%

%%%%% tiny citep
\newcommand{\tinycitep}[1]{%
    \bgroup
    \scriptsize
    \citep{#1}
    \egroup}
%%%%%

\begin{document}

\begin{frame}
    \titlepage
\end{frame}

\section{\textit{Megastudy} paradigma}

\begin{noheadline}

\begin{frame}
    \sectionpage
\end{frame}

\end{noheadline}

\subsection{Uvod}

\begin{frame}
    \frametitle{Uvod}

    \begin{itemize}
        \item tradicionalni način ispitivanja --- faktorijalni eksperimenti

        \pause

        \item \textit{megastudy}: velika istraživanja procesiranja jezika

        \pause

        \item prednosti pred faktorijalnim nacrtima
            \tinycitep{keuleersMegastudiesCrowdsourcingLarge2015,
            balotaMegastudiesWhatMillions2012}
            \begin{itemize}
                \item izjednačavanje podražaja na relevantnim dimenzijama
                \item korištenje kontinuiranih umjesto kategorijalnih mjera
            \end{itemize}
    \end{itemize}
\end{frame}

\begin{frame}
    \begin{itemize}
        \item ishod su velike, (često) otvorene baze podataka
            \begin{itemize}
                \item leksičke karakteristike (duljina riječi, učestalost
                    u jezičnim korpusima)
                \item psiholingvističke karakteristike (apstraktnost,
                    polisemija)
                \item mjere procesiranja (vrijeme reakcije, trajanje fiksacija)
            \end{itemize}

        \pause

        \item omogućavaju provođenje virtualnih eksperimenata
            \tinycitep{kupermanVirtualExperimentsMegastudies2015}

        \pause

        \item olakšavaju testiranje novih leksičkih i psiholingvističkih
            varijabli \tinycitep{yarkoniMovingColtheartNew2008}
    \end{itemize}
\end{frame}

\subsection[Što nam megastudije govore?]%
    {Što nam megastudije govore o procesiranju jezika?}

\begin{frame}
    \frametitle{Što nam megastudije govore o jeziku?}

    \begin{itemize}
        \item varijabla koja objašnjava uvjerljivo najviše varijance je
            učestalost riječi u jeziku
            \bgroup
            \scriptsize
            (\citealp*{balotaVisualWordRecognition2006};
            \citealp[ch. 6]{harleyPsychologyLanguageData2014})
            \egroup
        \begin{itemize}
            \pause

            \item kako definirati i mjeriti učestalost?

            \pause

            \item subjektivna učestalost
        \end{itemize}

        \pause

        \item duljina riječi \tinycitep{ferrandMEGALEXMegastudyVisual2018,
            brysbaertImpactWordPrevalence2016}

        \pause

        \item ortografska sličnost ili gustoća ortografskog susjedstva
            \tinycitep{coltheartAccessInternalLexicon1977,
            yarkoniMovingColtheartNew2008}
    \end{itemize}
\end{frame}

\begin{frame}
    \begin{itemize}
        \item vrijedni resursi za proučavanje jezika

        \begin{itemize}
            \item generalizacija nalaza na cjelokupni jezik
                \tinycitep{yarkoniGeneralizabilityCrisis2019}

            \pause

            \item ,,prototipno'' testiranje hipoteza

            \pause

            \item preciznija procjena parametara i prijatelji
                \tinycitep{yarkoniChoosingPredictionExplanation2017}
        \end{itemize}

    \end{itemize}
\end{frame}

\subsection{Što nam megastudije ne govore?}

\begin{frame}
    \frametitle{Što nam megastudije ne govore o jeziku?}

    \hspace*{\fill}
    \raisebox{37.4pt}{
        \begin{minipage}[t]{1em}
            \fontsize{56}{66}\selectfont
            \bfseries
            ,,
        \end{minipage}
    }
    \begin{minipage}{24em}
        In certain circles there is an almost religious faith that we can find
        the answers to [causal] questions in the data itself, if only we are
        sufficiently clever at data mining. However, [...] causal questions
        can never be answered from data alone.

        \bigskip

        \raggedleft
        \scriptsize
        \citet[str. 351]{pearlBookWhyNew2018}
    \end{minipage}
    \hspace*{\fill}

\end{frame}

\begin{frame}
    \begin{itemize}
        \item veliki setovi podatka ne mogu nadomjestiti reprezentativnost
            uzorka \tinycitep{mengStatisticalParadisesParadoxes2018}

        \pause

        \item standardni statistički modeli ne mogu zamijeniti znanstvenu
            ekspertizu
            \bgroup
            \scriptsize
            \citep[str. 526]{mcelreathStatisticalRethinkingBayesian2020}
            \egroup

        \pause

        \item važnost razvoja formalnih teorija
            \tinycitep{fiedlerToolsToysTruisms2004,
            navarroIfMathematicalPsychology2020,navarroDevilDeepBlue2019}
    \end{itemize}
\end{frame}

\section{Modeli procesiranja jezika}

\begin{noheadline}
    \begin{frame}
        \sectionpage
    \end{frame}
\end{noheadline}

\subsection{Drift-diffusion model}

\begin{frame}
    \frametitle{Drift-diffusion model \citep{ratcliffTheoryMemoryRetrieval1978}}

    \begin{itemize}
        \item zamišljen kao okvir koji može obuhvatiti različite paradigme u
            kognitivnoj psihologiji

        \pause

        \item primijenjen i na zadatak leksičke odluke
            \begin{itemize}
                \item do odluke se dolazi kroz proces koji tijekom vremena
                    prikuplja nejasne (\textit{noisy}) informacije

                \item informacije se prikupljaju sve dok se ne dosegne jedan od
                    kriterija za odlučivanje --- niz znakova je ili nije riječ
            \end{itemize}

        \pause

        \item uspješno modelira i prosječna vremena reakcija i njihove
            distribucije

        \item \citet{ratcliffDiffusionModelAccount2004} pokazali su osjetljivost
            modela na frekvencije riječi
    \end{itemize}
\end{frame}

\subsection{Bayesian reader}

\begin{frame}
    \frametitle{Bayesian reader \citep{norrisBayesianReaderExplaining2006}}

    \begin{itemize}
        \item specifično za čitanje

        \pause

        \item kreće od pretpostavke o idealnom opažaču koji donosi optimalne
            odluke

        \pause

        \item optimalno ponašanje ovisi o konkretnom zadatku koji opažač treba
            obaviti

        \item podaci koje opažač dobiva u sebi imaju šum, zbog čega mogu nastati
            pogreške u izvršavanju zadatka

        \pause

        \item uspješno lovi učinak frekvencije riječi, te učinak ortografske
            sličnosti (?)
    \end{itemize}
\end{frame}

\subsection{Modeli i megastudije}

\begin{frame}
    \frametitle{Modeli procesiranja jezika i megastudije}

    \begin{itemize}
        \item obilje otvorenih podataka dobivenih u megastudijama pogodno za
            testiranje različitih modela

            \begin{itemize}
                \item kros-validacija

                \item procjena parametara

                \item brzo testiranje prototipova i modifikacija modela

                \item testiranje modela na različitim jezicima
            \end{itemize}

    \end{itemize}
\end{frame}

\section{\textit{Multiverse} analiza}

\begin{noheadline}
    \begin{frame}
        \sectionpage
    \end{frame}
\end{noheadline}

\subsection{Od QRP do multiverzuma}

\setbeamertemplate{enumerate item}{\Roman{enumi})}

\begin{frame}
    \frametitle{Od QRP do multiverzuma}

    \begin{enumerate}
        \item QRP --- eng. \textit{questionable research practices}
            \bgroup
            \scriptsize
            \citep*{johnMeasuringPrevalenceQuestionable2012}
            \egroup

        \pause

        \begin{itemize}
            \item nisu naveli sve zavisne varijable
            
            \item nisu naveli sve eksperimentalne situacije

            \item isključivanje podataka nakon što su pogledali kako to utječe
                na rezulate
        \end{itemize}

        \pause

        \item mnogo različitih načina na koje se podaci \emph{mogu analizirati}
            \tinycitep{gelmanGardenForkingPaths2013}

            \begin{itemize}
                \item ,,[R]esearchers can perform a reasonable analysis
                    given their assumptions and their data, but had the data
                    turned out differently, they could have done other analyses
                    that were just as reasonable in those circumstances''
                    (str. 1)

                \pause

                \item povezanost statističkih i znanstvenih hipoteza
                    \bgroup
                    \scriptsize
                    \citep[također,][]{yarkoniGeneralizabilityCrisis2019}
                    \egroup
            \end{itemize}

        \seti

    \end{enumerate}
\end{frame}

\begin{frame}
    \begin{enumerate}
        \conti

        \item mnogo različitih načina na koje se podaci \emph{analiziraju}
            \tinycitep{silberzahnManyAnalystsOne2018}

            \begin{itemize}
                \item pitanje: jesu li nogometni suci skloniji crvene kartone
                    davati crnim igračima?

                \pause

                \item 29 istraživačkih timova analiziralo je isti set
                    podataka

                \pause

                \item 29 različitih načina na koje su podaci analizirani

                \pause

                \item ,,dva'' različita zaključka
            \end{itemize}

    \end{enumerate}
\end{frame}

\subsection{\textit{Multiverse} analize}

\begin{frame}
    \frametitle{IV) \textit{Multiverse} analize}

    \begin{itemize}
        \item ,,A multiverse analysis starts from the observation that data used
            in an analysis are usually not just passively recorded in an
            experiment or an observational study.  Rather, data are to a certain
            extent actively constructed.''
            \bgroup
            \scriptsize
            (\citealp*[str. 702]{steegenIncreasingTransparencyMultiverse2016};
            \citealp*[za sličan pristup
            vidi][]{simonsohnSpecificationCurveDescriptive2015})
            \egroup

        \pause

        \item domisliti velik broj mogućih specifikacija problema

        \item provesti analizu uz svaku moguću specifikaciju

    \end{itemize}

\end{frame}

\begin{frame}
    \centering
    \includegraphics[scale=.25]{orben.png}

    \tinycitep{orbenAssociationAdolescentWellbeing2019}
\end{frame}

\subsection{\textit{Multiverse megastudy}}

\begin{frame}
    \frametitle{\textit{Multiverse} analiza i megastudije}

    \begin{itemize}
        \item pretprocesiranje podataka
            \begin{itemize}
                \item odbaciti vremena reakcija van \(\pm 2 SD\)? \(3 SD\)?
                \item koje sudionike zadržati na temelju točnosti odgovora?
                    \(>80\%? >90\%\)?
            \end{itemize}

        \pause

        \item kako modelirati vremena reakcija?
            \begin{itemize}
                \item različiti statistički modeli

                \pause

                \item koje mjere centralne tendencije koristiti?
                    \tinycitep{rousseletReactionTimesOther2020}

                \pause

                \item kako i koju distribuciju modelirati?
                    \tinycitep{balotaMovingMeanStudies2011}

            \end{itemize}
    \end{itemize}
\end{frame}

\section{Rasprava}

\begin{noheadline}

\begin{frame}
    \sectionpage
\end{frame}

\end{noheadline}

\begin{frame}
    \begin{center}
        \begin{minipage}{25em}
            O čemu pričamo kad pričamo o procesiranju jezika?
            \tinycitep{elliottWhatTestRetestReliability2020}
        \end{minipage}
    \end{center}
\end{frame}

\begin{frame}
    \hspace*{\fill}
    \raisebox{50.5pt}{
        \begin{minipage}[t]{1em}
            \fontsize{56}{66}\selectfont
            \bfseries
            ,,
        \end{minipage}
    }
    \begin{minipage}{24em}
        The empiricist period [...] is characterized by a general
        lack of interest in stringent theoretical and metatheoretical thinking.
        [...]  Industrious research appears to be mainly driven by available
        research tools [...], methods of analysis [...], and the dynamics
        of scientific subcommunities but is surprisingly often detached from
        clearly spelled-out theories.

        \bigskip
        \scriptsize
        \raggedleft
        \citep[str. 123]{fiedlerToolsToysTruisms2004}
    \end{minipage}
    \hspace*{\fill}
\end{frame}

\begin{frame}
    \hspace*{\fill}
    \raisebox{74.5pt}{
        \begin{minipage}[t]{1em}
            \fontsize{56}{66}\selectfont
            \bfseries
            ,,
        \end{minipage}
    }
    \begin{minipage}{24em}
        There is a one-to-many mapping from scientific to statistical
        hypotheses.

        \begin{flushright}
            \bigskip
            \scriptsize
            \citep[str. 6]{gelmanGardenForkingPaths2013}
        \end{flushright}

        \bigskip

        [A] huge proportion of the verbal claims made in empirical psychology
        articles turn out to have very little relationship with statistical
        quantities they putatively draw their support from.

        \begin{flushright}
            \bigskip
            \scriptsize
            \raggedleft
            \citep[str. 2]{yarkoniGeneralizabilityCrisis2019}
        \end{flushright}
    \end{minipage}
    \hspace*{\fill}
\end{frame}

\setbeamertemplate{headline}{}

\begin{frame}[allowframebreaks]
    \frametitle{Reference}

    \bibliographystyle{apacite}
    \bibliography{../reference}
\end{frame}

\end{document}
